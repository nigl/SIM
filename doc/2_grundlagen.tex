\chapter{Grundlagen}

\section{Nagelschreckenberg Model}
TODO beschreiben des nagelschreckenberg models.
Also ablauf und evtl. auch schon das Fundamentaldiagram.

TODO das hier anpassen...

\begin{algorithm}[H]
 %\SetLine % For v3.9
 %\SetAlgoLined % For previous releases [?]
 \KwData{ \\
   Seitenlänge quadratisches Eingabebild: \(n\)\\
   RGB-Werte Eingabebild: \(r(x,y), \; b(x,y), \; g(x,y) \quad x,y \in \{0, \dots, n-1\} \) 
 }
 \KwResult{ \\
   RGB-Werte neues Bild: \(\overline{r}(x,y), \; \overline{b}(x,y), \; \overline{g}(x,y) \quad x,y \in \{0, \dots, n-1\} \)
 }

 %initialization\;
 \For{ \(k \in \{1, \cdots, n\}\) }{

	\For{ \(j \in \{1, \cdots, n\}\) }
	{
    \mbox{}

    \(\overline{x} = 2k+j \; \modk\);

    \(\overline{y} = k+j \; \modk\);
    
    \mbox{}

    \(\overline{r}(\overline{x}, \overline{y}) =  r(k,j)\);

    \(\overline{g}(\overline{x}, \overline{y}) =  g(k,j)\);

    \(\overline{b}(\overline{x}, \overline{y}) =  b(k,j)\);
	}

 }
 \caption{Nagelschreckenberg Algorithmus}
 \label{algo:nagelsberg}
\end{algorithm}

\section{Erweitertes Model für Kreuzungen} \label{sec:erwmodel}
TODO unser model erklären (autos bremsen vor kreuzung immer ab,
... etc)
