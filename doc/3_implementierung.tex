\chapter{Simulationsprogramm}
Zur Bewältigung der in Kapitel \ref{sec:aufgabenstellung} beschriebenen
Aufgaben wurde das erweiterte Nagelschreckenberg Model \see{sec:erwmodel}
durch ein Matlabskript umgesetzt. Im folgenden wird dieses
Programm besprochen.

\section{Programmbedienung}
TODO beschreibung der Programmfunktionalität (buttons anzeigen etc.)

\section{Implementierung}
Von Beginn an wurde bei der Umsetzung darauf geachtet, wo immer es möglich 
war, auf Schleifen zu verzichten und stattdessen die in Matlab sehr effizient 
umgesetzten Matrixoperationen zu benutzen. Dies bedeutet vor allem, dass als 
grundlegende Datenstruktur Matrizen zu benutzen sind. 

Es besteht eine Kreuzung aus zwei Straßen, welche zur Vereinfachung als Horizontale und Vertikale
bezeichnet werden. Beide Straßen haben den den Kreuzungspunkt als gemeinsamen Straßenabschnitt. Dieser
wird zweimal, in der Horizontalen und in der Vertikalen, abgespeichert.

Eine Straße wird durch zwei Matrizen \(S^{1}, S^{2}\) repräsentiert. 
Die \(t\)-te Zeile dieser Matrizen stellt die Situation zur Zeit \(t\) (mit \(1 \leq t \leq T\)) dar, 
wobei die Spalten den Zellen entsprechen in denen ein Auto stehen kann. 
Steht also zur Zeit \(t\) das Auto mit Nummer \(k\) (\(1 \leq k \leq K\)) und Geschwindigkeit \(v_k\) in der \(n\)-ten Zelle, 
so so gilt \( S^{2}_{t, n} = k\) und \(S^{1}_{t, n} = v_k\). 
Ist andererseits zur Zeit \(t\) die \(n\)-te Zelle nicht belegt so gilt: \(S^2_{t, n} = 0 = S^1_{t, n}\).
Kurz gesagt \(S^{1}\) die Geschwindigkeiten und \(S^{2}\) die Autonummern zu jeder stelle und jedem Zeitpunkt.
