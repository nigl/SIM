\chapter{Simulationsprogramm}
Zur Bewältigung der in Kapitel \ref{sec:aufgabenstellung} beschriebenen
Aufgaben wurde das erweiterte Nagelschreckenberg Model \see{sec:erwmodel}
durch ein Matlabskript umgesetzt. Im folgenden wird dieses
Programm besprochen.

\section{Programmbedienung}
TODO beschreibung der Programmfunktionalität (buttons anzeigen etc.)

\section{Implementierung}
Von Beginn an wurde bei der Umsetzung darauf geachtet, wo immer es möglich 
war, auf Schleifen zu verzichten und stattdessen die in Matlab sehr effizient 
umgesetzten Matrixoperationen zu benutzen. Dies bedeutet vor allem, dass als 
grundlegende Datenstruktur Matrizen zu benutzen sind. 

\subsection{Kreuzung, Straße und Auto}
Eine Kreuzung besteht aus zwei Straßen, welche zur Vereinfachung als Horizontale und Vertikale
bezeichnet werden. Beide Straßen haben den Kreuzungspunkt als gemeinsamen Straßenabschnitt. Dieser
wird zweimal, in der Horizontalen und in der Vertikalen, abgespeichert. 

Weiter muss die Verkehrssitation für jeden Zeitpunkt abgespeichert werden und
es soll möglich sein Hinderenisse einzubauen, an dennen Autos langsam vorbei Fahren müssen, 
wie z.B. Baustellen oder eben die Kreuzung selbst. 
Für eine Straße werden demnach die folgenden Daten benötigt:
\begin{enumerate}
  \item Anzahl der Zellen \(N \in \mathbb{N}\)
  \item Anzahl der Zeitschritte die Simuliert werden \(T \in \mathbb{N}\)
  \item Koordinate des Kreuzungspunktes \(c \in \{ 1, \ldots N \}\)
  \item Koordinaten der Hindernisse \(o_i \in \mathbb{N}, \; i = 1, \ldots l\)
\end{enumerate}
Die Koordinaten entsprechen dabei den Zellennummern. Insbesondere ist eine Kreuzung ein spezielles Hinderniss,
so dass sich Autos langsam dieser nähern. 
Was noch fehlt sind die Fahrzeuge selbst. Diese werden zu einem festen Zeitschritt beschrieben druch:
\begin{enumerate}
  \item Autonummer \(k \in \{ 1, \ldots, K \}, \; K \leq N\) 
  \item Geschwindigkeit \(v_k \in \{0, \ldots, v_{max} \}\)
  \item Position, d.h. Straße und Zellenkoordinate
\end{enumerate}
Dabei ist die Autonummer nur innerhalb einer Straße eindeutig. Dies ist ausreichend, da die Fahrzeuge nicht 
abbiegen, d.h. eine Straße nicht verlassen.

Insgesamt wird eine Straße durch zwei \(T \times N\)-Matrizen \(V, W\) beschrieben.
Die \(t\)-te Zeile dieser Matrizen stellt die Situation zur Zeit \(t\) (mit \(1 \leq t \leq T\)) dar, 
wobei die Spalten den Zellen entsprechen in denen ein Auto stehen kann. 
Steht also zur Zeit \(t\) das Auto mit Nummer \(k\) (mit \(1 \leq k \leq K\)) und Geschwindigkeit \(v_k\) in der \(n\)-ten Zelle, 
so so gilt \( W_{t, n} = k\) und \(V_{t, n} = v_k\). 

Ist andererseits zur Zeit \(t\) die \(n\)-te Zelle nicht belegt so gilt: \(W_{t, n} = 0 = V_{t, n}\).
Kurz gesagt \(V\) speichert die Geschwindigkeiten und \(W\) die Autonummern zu jeder Stelle und jedem Zeitpunkt.

Zu Beginn der ist nur die jeweils erste Zeile, d.h. der erste Zeitschritt, initialisiert die anderen Matrixeinträge
werden mit Null vorbelegt. Dabei werden die Autos mit zufälliger Position und Geschwindigkeit vorinitialisiert, gemäß
einer eingestellten dichte. Dies erledigt die Funktion init\_street.m .
