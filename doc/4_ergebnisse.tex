\chapter{Validierung}
TODO bildchen uns zeug (Plot des fundamental diagramms für
verschiedene dichteeinstellungen etc...)

TODO Plots im CipPool fertigen 

Für jede Simulation ist von großer Bedeutung, wie weit Ergebnisse mit der Realität übereinstimmen, deshalb möchten wir dies im folgenden Abschnitt näher untersuchen. Dabei wollen wir auf zwei mögliche Fehlerursachen eingehen, zum einen die Richtigkeit der Modellierung und zum Anderen die Umsetzung der Implementierung. 

Um unsere Implementierung zu prüfen, haben wir mehrere Faktoren betrachtet. Viele Fehler kann man bereits bei der Betrachtung der Animation sehen, bewegen sich die Autos wie erwartet, oder gibt es Unregelmäßigkeiten. So konnten viele Fehler während der Entwicklungsphase entdeckt und anschließend ausgeräumt werden. Natürlich gibt es auch Fehler die nicht so einfach gefunden werden können, daher haben wir einige Tests durchgeführt. Unter Anderem haben wir geprüft, ob die Anzahl der Autos während einer Simulation konstant bleibt, dass also keine Autos verschwinden oder plötzlich auftauchen. Auch haben wir dank der Mitspeicherung der Autonummer sehen können, dass die Reihenfolge immer gleich bleibt. 

Ein wichtiges Tool für die Validierung ist das Fundamentaldiagramm, es stellt den Zusammenhang zwischen der Dichte und des Verkehrsflusses dar. Da bei unserer Kreuzung der Fluss hauptsächlich von der Dichte auf der vertikalen Ringstraße abhängt, ist bei unseren Dichtediagrammen immer die vertikale Dichte auf der $x$-Achse angetragen. 

Um das Fundamentaldiagramm zu erstellen, muss man den Verkehrsfluss $f$ und die Verkehrsdichte $\rho$ bestimmen. Der Verkehrsfluss $f$ an einem bestimmten Messpunkt ist bestimmt durch
\[ f = \frac{P}{\delta T}. \]
Wobei $P$ die Anzahl der Autos, die den Messpunkt während des Zeitintervalls $\delta T$ passiert haben. In unserem Fall haben wir dieses Intervall $\delta T$ auf die gesamte Simulationsdauer (ungefähr $100$ Sekunden) festgelegt. Als Messpunkt haben wir uns auf die Kreuzung festgelegt, da diese Stelle am interessantesten ist. Für die Dichte haben wir die gesamte Dichte auf der vertikalen Ringstraße verwendet, dies kann durch folgende Formel
\[ \rho = \frac{N}{L} \]
bestimmt werden. $N$ ist hierbei die konstante Anzahl der Autos auf der vertikalen Straße und $L$ ist die totale Länge Ringstraße. Diese Vorgehensweise ist auch in \cite{book:bungartz} Abschnitt 8.3.3 zu finden, allerdings haben wir einen Spezialfall davon verwendet, da bei uns die Variable $L$ gleich der gesamten Straßenlänge ist.

\begin{figure}%
\centering
\includegraphics[width=7cm]{4_BestFD.png}%
\caption{Bestimmung des Verkehrsflusses und der Verkehrsdichte, in unserem Fall ist $\delta T$ gleich der gesamten Simulationsschritte und $L$ ist die gesamte Straßenlänge. Quelle: \cite{book:bungartz} Abb. 8.8}%
\label{pic:FD_Skizze}%
\end{figure}

Zunächst konnten wir unsere Ergebnisse mit den Resultaten aus \cite{book:bungartz} für eine einfache Ringstraße vergleichen. Dafür haben wir die Kreuzungsdaten aus unserem Tool entfernt, dadurch simulieren wir eine einfache Ringstraße ohne Hindernisse. In der Abbildung \ref{pic:FD_Vergleich} werden die unterschiedlichen Diagramme dargestellt. Aufgrund der hohen Ähnlichkeit, können wir grobe Implementierungsfehler im Grundgerüst unseres Programmes ausschließen.

\begin{figure}%
\centering
\includegraphics[width=12cm]{4_FD_Vergleich.png}%
\caption{Oben: Unsere erzeugten Diagramme Unten: Diagramme aus \cite{book:bungartz} Abb. 8.9. Jeweils mit Trödelwahrscheinlichkeit $p=0.2$}%
\label{pic:FD_Vergleich}%
\end{figure}





Maybe TODO Erklären warum Stauwellen immer gleiche Signalgeschwindigkeit haben, siehe Bungartz Seite 178

