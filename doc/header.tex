\documentclass[pagesize=pdftex,paper=a4,bibliography=totoc,listof=totoc,12pt]{scrreprt}

%%%%%%%%%%%%%%%%%DATEN%%%%%%%%%%%%%%%%%
%Erstellungsdatum
\date{\today}

%%%%%%%%%%%%%%%%%pakete%%%%%%%%%%%%%%%%%
% böses hackpaket
\usepackage{scrhack}

%neue Rechtschreibung
\usepackage[ngerman]{babel}
\usepackage{babelbib}
 
%\usepackage[T1]{fontenc}
%Umlaute ermölichen utf8 ( latin1)
\usepackage[utf8]{inputenc}
 
% Tabellen
\usepackage{array}
 
% Schriftfarben
\usepackage{color}

% FloatBarrier
\usepackage{placeins}
%Bilder
\usepackage{float}
\usepackage{floatflt}
\usepackage[pdftex]{graphicx}
\usepackage{caption} %für subfigure
\usepackage{subcaption} % für subfigure
\DeclareGraphicsRule{*}{mps}{*}{}
%\usepackage{PSTricks}

%Math
\usepackage{amsmath}
% bessere theorem umgebung als amsthm (zeilen umbruch nach überschritf und aufzählung)
\usepackage[amsmath, thmmarks]{ntheorem} 
\usepackage{amssymb}
\usepackage{mathrsfs} % \mathscr command for categories
\usepackage[all]{xy} % Kommutative-Diagramme

% Pseudocode in latex
\usepackage{algorithm2e}

%Nomenclature
\usepackage[intoc]{nomencl}

%Kopf-/Fußzeilen
\usepackage{fancyhdr}

%Hyperref für Bookmarks im PDF
\usepackage[pdftex, pagebackref]{hyperref}

%%%%%%%%%%%%%%%%%%%%seitencfg%%%%%%%%%%%%%%%%%%%%%%%%%%%%
\pagestyle{fancy}
\fancyhf{}
%Kopfzeile links bzw. innen
\fancyhead[L]{\nouppercase{\leftmark}}
%Kopfzeile rechts bzw. außen
\fancyhead[R]{\thepage}
%Linie open
\renewcommand{\headrulewidth}{0.5pt}
%Fußzeile links bzw. innen
\fancyfoot[L]{Hausarbeit Simulation}
%Fußzeile rechts bzw. außen
\fancyfoot[R]{\today}
%Linie unten
\renewcommand{\footrulewidth}{0.5pt}

\hypersetup {
pdftitle = {Verkehrsflusssimulation einer Kreuzung},
pdfsubject = {Hausarbeit Simulation}, 
pdfauthor = {Julian Berndt, Hannah Dusch, Martin Kraus, Philipp Schwarz},
pdfhighlight = {/O},
pdfkeywords = {Nagelschreckenberg, Kreuzung, Verkehrssimulation, Fundamentaldiagramm}, colorlinks = {false},
bookmarksnumbered = {true},
citebordercolor = {1 1 1},
linkbordercolor = {1 1 1},
urlbordercolor = {1 1 1},
bookmarksopen = {true},
bookmarksopenlevel = {1}
}

% Nummerierungen bei Aufzählungen (Erste Ebene Römisch) 
\renewcommand{\theenumi}{\roman{enumi}}
\renewcommand{\labelenumi}{\theenumi)}

%%%%%%%%%%%%%%%%%%%EIGENE-BEFEHLE%%%%%%%%%%%%%%%%%%%%%%%%
\newcommand{\ncl}[2]{\nomenclature{\textit{#1}}{\textcolor{white}{test}\\#2}}
\newcommand{\glossar}[1]{\textit{#1}\glossary{\textit{#1}}}

%Bilder
\newcommand{\pic}[1]{Abbildung \ref{#1}}
\newcommand{\picp}[1]{Abbildung \ref{#1} Seite \pageref{#1}}
%Tabellen
\newcommand{\tab}[1]{Tabelle \ref{#1}}
\newcommand{\tabp}[1]{Tabelle \ref{#1} Seite \pageref{#1}}
%Darstellung eines Links im Quellenverzeichnis
\newcommand{\urlg}[2]{\url{#1}\\zuletzt abgerufen am #2}
%Verweise
\newcommand{\see}[1]{(siehe Kapitel \ref{#1})}
\newcommand{\seeo}[1]{siehe Kapitel \ref{#1}}
\newcommand{\seea}[1]{(siehe Anhang \ref{#1})}
\newcommand{\refThmIt}[1]{\textit{\ref{#1})}}
% Fett und Schief
\newcommand{\textbit}[1]{\textbf{\textit{#1}}}
%\newcommand{\beweis}{\textbf{Beweis:} }
%\newcommand{\qed}{\hfill \(\square\)}
%\newcommand{\noteH}{\textbf{Bemerkung:} }
\newcommand{\defH}[1]{\underline{#1}}

% theoremartige Konstrukte
%-------------------------
% diese haben keinen zeilenumbruch nach d. Überschrift
\theoremstyle{plain}
\newtheorem{Def}{Definition}[chapter]
\newtheorem{Satz}{Satz}[chapter]
\newtheorem{Korollar}{Korollar}[chapter]
\newtheorem{Lemma}{Lemma}[chapter]

% Jetzt mit Zeilen umbruch nach Überschrift 
% (Counter aber von obigen Konstrukten benutzen)
\theoremstyle{break}
\newtheorem{Def-break}[Def]{Definition}
\newtheorem{Satz-break}[Satz]{Satz}
\newtheorem{Korollar-break}[Korollar]{Korollar}
\newtheorem{Lemma-break}[Lemma]{Lemma}

% Bemerkungen
\theoremstyle{plain}
\theorembodyfont{\normalfont}
\theoremseparator{ :}
\newtheorem{Bem}{Bemerkung}[chapter]

% Beispiele
\theoremstyle{plain}
\theorembodyfont{\normalfont}
\theoremseparator{ :}
\newtheorem{Bsp}{Beispiel}[chapter]

% Eigene beweisumgebung
\theoremstyle{nonumberplain}
\theoremheaderfont{\itshape}
\theorembodyfont{\normalfont}
\theoremseparator{.}
\theoremsymbol{\ensuremath{\square}}
\newtheorem{proof}{Beweis}
\newtheorem{proofsketch}{Beweisskizze}
% Fix für die Beweisumgebung:
% Steht vor \end{proof} ein \end{align}, so wird das 
% qed-symbol nicht so schön gesetzt: \mbox{} einfügen
%-----------------------------

% mathebefehle
%-------------
\DeclareMathOperator{\Hom}{Hom} % Klasse der Morphismen einer Kategorie
\DeclareMathOperator{\Morph}{Morph} % Klasse der Morphismen einer Kategorie
\DeclareMathOperator{\Obj}{Ob} % Klasse der Objekte in der Kategorie
\DeclareMathOperator{\id}{id} % Identische Abbildung / Morphismus
\DeclareMathOperator{\mod1}{(mod\mbox{ }1)} % mein Modulo-Operator
\DeclareMathOperator{\modk}{(mod\mbox{ }n)} % mein Modulo-Operator
\DeclareMathOperator{\diverg}{div} % Divergenz
\newcommand*\rfrac[2]{{}^{#1}\!/_{#2}} % schönere Brüche (Schiefes bruchzeichen)
\newcommand{\sprod}[2]{ \left\langle #1 , #2 \right\rangle } % skalarprodukt
\newcommand{\dcup}{\mathbin{\dot{\cup}}} % Disjunkte Vereinigung
\newcommand{\dbigcup}{\mathbin{\dot{\bigcup}}} % Disjunkte Vereinigung
% Kategorien
\newcommand{\Ens}{{\mathscr{E}\!ns}} % Kategorie der Mengen
\newcommand{\Top}{{\mathscr{T}op}} % Kategorie der topologischen Räume
\newcommand{\Mess}{{\mathscr{M}\!ess}} % Kategorie der Messräume
\newcommand{\Mass}{{\mathscr{M}\!ass}} % Kategorie der Messräume
\newcommand{\Ban}{{\mathscr{B}an}} % Kategorie der Bannachräume
\newcommand{\Hil}{{\mathscr{H}il}} % Kategorie der Hilberträume
\newcommand{\Dyn}{{\mathscr{D}yn}} % Kategorie der dynamischen Systeme
\newcommand{\Mdyn}{{\mathscr{M}\!dyn}} % Kategorie der maßtheoretischen dynamischen Systeme
\newcommand{\Tdyn}{{\mathscr{T}\!dyn}} % Kategorie der topologischen dynamischen Systeme
%\newcommand{\Ban1}{{\mathscr{B}an_1}} % Kategorie der Banachräume (Isometrien)

%\newcommand{\comp
%\newcommand{\comp}[1]{{#1}^{c}} % Mengenkomplement
%\newcommand{\compm}[2]{{#1}^{c(#2)}} % Mengenkomplement mit Grundmenge
%\newcommand{\compm}[2]{{#1}^{c\mathsf{[}#2\mathsf{]}}} % Mengenkomplement mit Grundmenge
%\newcommand{\compm}[2]{{#1}^{\complement\tiny{#2}}} % Mengenkomplement mit Grundmenge

%%%%%%%%%%%%%%%%%%%%NOMENCLATURE/GLOSSAR%%%%%%%%%%%%%%%%%%%
%Nomenclature 
\makenomenclature

%Glossar
%\include{glossar}

%%chemie
%\usepackage[version=3]{mhchem}

%own commands
%Merkbox:
%\begin{merkbox}
% text
%nomenclature}
% \newsavebox\TBox
% \newenvironment{merkbox}
%{%\par\noindent
% \begin{lrbox}{\TBox}
% \varwidth{\textwidth-2.5\fboxsep}
% }{\endvarwidth\end{lrbox}%
% \textcolor{orange}{\textbf{Merke:}}\\[1mm]
% \Ovalbox{\usebox\TBox}\par
%}

%markierungen
%\newcommand{\texthigh}[1]{\textcolor{orange}\textbf{#1}}
%\newcommand{\texthead}[1]{\emph{#1:}\hline}
