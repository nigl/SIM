\chapter{Fazit}

Fazit: Ziel unserer Simulation war es, Hinweise auf den maximalen Verkehrsfluss auf beiden Straßen zu bestimmen. Dazu wollen wir den gesamten Fluss bestimmen, dies ist einfach die Summe aus den beiden einzelnen Verkehrsflüssen.\\

Der Verkehrsfluss der horizontalen Straße hängt im wesentlichen von der Dichte der vertikalen Straße ab. Aus \cite{book:bungartz} ist bekannt, dass für eine einfache Ringstraße der Verkehrsfluss für eine Dichte von $\rho_max \approx 0.12$ maximal ist, dabei ist der Fluss $f \approx 3000$ Autos pro Stunde. Außerdem können wir aus dieser Quelle entnehmen, dass realistische Verkehrsbedingungen mit einer Trödelwahrscheinlickeit von $p=0.2$ erreicht werden. Deshalb haben wir im Folgenden die horizontale Dichte $\rho_v$ auf den Wert $0.12$ festgesetzt und sind von einer Trödelwahrscheinlichkeit $0.2$ ausgegangen. 
%
\begin{figure}[h]%
\centering
\includegraphics[width=17cm]{MaxFluss.png}%
\caption{Die verschiedenen Flüsse für $p=0.2$, $rho_h=0.12$ und einer Straßenlänge von je 300 Zellen.  }%
\label{pic:MaxFluss}%
\end{figure}
%
In der Abbildung \ref{pic:MaxFluss} sind die einzelnen Flüsse in einer Grafik dargestellt. Man kann erkennen, dass sich der gesamte Fluss im wesentlichen Konstant auf einem Niveau von ca. 1500 Autos pro Stunde hält. Ein Grund für diese Erscheinung ist die Bedingung, dass sich beide Ringstraßen eine Zelle, nämlich die Kreuzungszelle teilen. Dadurch wird der Fluss begrenzt, da jedes Auto vor der Kreuzung auf Geschwindigkeit 1 abbremst und somit jedes Auto auf der Kreuzungszelle gewesen sein muss. Durch das Abbremsen wird auch die Geschwindigkeit auf der Kreuzung limitiert, so kann ein Auto nur mit geringer Geschwindigkeit über die Kreuzung fahren.

Ausblick:\\
TODO mögliche Erweiterungen wie Ampel oder mehr Kreuzungen oder Abbiegen etc.
